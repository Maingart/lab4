\def \equationFirst {y'' + y' - 4y = 0}

\def \equationSecond {y'' - 2y' + 10y = 5x^2 \cdot e^x \cdot \cos{3x}}

\def \equationThird {x^2 \cdot (2\ln{x} - 1) \cdot y'' + 4y = x \cdot (2\ln{x} + 1) \cdot y'}

\def \equationFourth {x(y'' - y)\sin{x} + 2(xy' + y)\cos{x} + 2y'\sin{x} = e^x}

\def \equationFifth {(x^2 + 1)y'' = 2y}

\newcommand{\solutionItemFirst}[3]{
    \item 
    	$ #1 $
    	
    \begin{tabular}[t]{l p{10.6cm}}
    	\textit{Характеристика:} & #2 \\
    	\textit{Общее решение:} & $ #3 $ \\
    \end{tabular}
}


\section{Задача 1.}
\subsection{Постановка задачи}
Для следующих линейных дифференциальных уравнений 
дать характеристику и найти общее решение:


\begin{enumerate}
    \item $ \equationFirst $
    \item $ \equationSecond $
    \item $ \equationThird $
    \item $ \equationFourth $
    \item $ \equationFifth $
\end{enumerate}

\subsection{Решение}
\begin{enumerate}
    \solutionItemFirst
    	{\equationFirst}
    	{
    		Линейное однородное дифференциальное уравнение второго порядка 
    		с постоянными коэффициентами
    	}
    	{
    		y = C_1e^{\left( \tfrac{\sqrt{17}}{2} - \tfrac{1}{2} \right)x } +
    			C_2e^{\left(-\tfrac{\sqrt{17}}{2} - \tfrac{1}{2} \right)x }
    	}
    
  	\clearpage
  	
  	\solutionItemFirst
	  	{\equationSecond}
	  	{
	  		Линейное неоднородное дифференциальное уравнение второго порядка 
	  		с постоянными коэффициентами
	  	}
	  	{
	  		y = C_1e^x\sin{3x} + C_2e^x\cos{3x} +
	  			\dfrac{5}{18} e^x x^3 \sin{3x} + 
	  			+ \hspace{0.25em} \dfrac{5}{36} e^x x^2 \cos{3x} -
	  			\dfrac{5}{108} e^x x \sin{3x}
	  	}
  	
	\vspace{1.5em}
  	
  	\solutionItemFirst
	  	{\equationThird}
	  	{
	  		Линейное однородное дифференциальное уравнение второго порядка
	  		с переменными коэффициентами
	  	}
	  	{
	  		y = C_1\ln{x}
	  		\left( 
	  			2\ln{\left( 2\ln{x} - 1 \right)} - 
	  			2\ln{\left(\ln{x}\right)} + \dfrac{1}{\ln{x}}
	  		\right) +
	  		+ \hspace{0.25em} C_2\ln{x}
	  	}
  	
	\vspace{1.5em}
  	
	\solutionItemFirst
		{\equationFourth}
		{
			Линейное неоднородное дифференциальное уравнение второго порядка, 
			допускающее интегрирование дважды
		}
		{
			y = \dfrac{C_1 \csc{x}}{x} + C_2\csc{x} + \dfrac{e^x\csc{x}}{x}
		}
	
	\vspace{1.5em}  	
	
  	\solutionItemFirst
	  	{\equationFifth}
	  	{
	  		Линейное однородное дифференциальное уравнение второго порядка 
	  		с переменными коэффициентами
	  	}
	  	{
	  		y = C_1 (( x^2 + 1 )\arctan{x} + x) + C_2(x^2 + 1)
	  	}
  
\end{enumerate}
